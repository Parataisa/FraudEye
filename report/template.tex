%%%%%%%%%%%%%%%%%%%%%%%%%%%%%%%%%%%%%%%%%%%%%%%%%%%%%%%%%%%%%%%%%%%%%%%%%%%%%%%%
%2345678901234567890123456789012345678901234567890123456789012345678901234567890
%        1         2         3         4         5         6         7         8

\documentclass[a4, 10 pt, conference]{ieeeconf}  % Comment this line out if you need a4paper

%\documentclass[a4paper, 10pt, conference]{ieeeconf}      % Use this line for a4 paper

\IEEEoverridecommandlockouts                              % This command is only needed if 
                                                          % you want to use the \thanks command

\overrideIEEEmargins                                      % Needed to meet printer requirements.

% See the \addtolength command later in the file to balance the column lengths
% on the last page of the document

% The following packages can be found on http:\\www.ctan.org
%\usepackage{graphics} % for pdf, bitmapped graphics files
%\usepackage{epsfig} % for postscript graphics files
%\usepackage{mathptmx} % assumes new font selection scheme installed
%\usepackage{times} % assumes new font selection scheme installed
%\usepackage{amsmath} % assumes amsmath package installed
%\usepackage{amssymb}  % assumes amsmath package installed
\usepackage{multicol}
\usepackage{tcolorbox}
\usepackage{cuted,tcolorbox,lipsum}
\usepackage{xcolor}
\usepackage{hyperref}

\title{\LARGE \bf
Introduction to Machine Learning (SS 2024)\\ Programming Project
\vspace{-3em}
}


%\author{Fabio Plunser$^{1}$ and Dominik Barbist$^{2}$% <-this % stops a space
%}


\begin{document}


\maketitle
\vspace{-3em}
\thispagestyle{empty}
\pagestyle{empty}

\begin{strip}
  \begin{tcolorbox}[
      size=tight,
      colback=white,
      boxrule=0.2mm,
      left=3mm,right=3mm, top=3mm, bottom=1mm
    ]
    {\begin{multicols}{2}% replace 3 with 2 for 2 authors.

        \textbf{Author 1}       \\
        Last name: Plunser              \\  % Enter first name
        First name: Fabio             \\  % Enter first name
        Matrikel Nr.:               \\  % Enter Matrikel number

        \columnbreak

        \textbf{Author 2}       \\
        Last name: Barbist              \\  % Enter first name
        First name: Dominik             \\  % Enter first name
        Matrikel Nr.:               \\  % Enter Matrikel number

        \columnbreak

      \end{multicols}}
  \end{tcolorbox}
\end{strip}

%%%%%%%%%%%%%%%%%%%%%%%%%%%%%%%%%%%%%%%%%%%%%%%%%%%%%%%%%%%%%%%%%%%%%%%%%%%%%%%%

\section{Introduction}
\label{sec:intro}
We selected the Transaction dataset, which includes approximately 227,000 data points, each having 30 features (1 Time feature, 1 Amount feature, and 28 anonymized features). The dataset is labeled such that a transaction is
marked with a 1 if it is fraudulent and 0 if it is not, making the task a binary classification problem.

The dataset is highly imbalanced, with only about 0.001729\% of the data points labeled as fraudulent. This extreme imbalance poses a significant challenge for the classification task, as the model may become
biased towards predicting the majority class (non-fraudulent transactions).

The 28 other features are anonymized, and their exact meanings are unknown. However, these features do not contain any missing values, ensuring that each feature is ready for use in
machine learning algorithms without additional preprocessing for imputation or scaling.

\section{Implementation / ML Process}
\label{sec:methods}

{\color{blue}

  \begin{itemize}
    \item Did you need to pre-process the dataset (e.g. augmenting data points, extracting features, reducing the dimensionality, etc.)? If so, describe how you did this.
    \item Specify the method (e.g. linear regression, or neural network, etc.). You do not have to describe the algorithm in detail, but rather the algorithm family and the properties of the algorithm within that family, e.g. which distance functions for a decision tree, what architecture (layers and activations) for a neural network, etc.
    \item State (in 2-5 lines) what makes the algorithm you chose suitable for this problem. What are the reasons for choosing your ML method over others?
    \item If you used a method that was not covered in the VO, describe how it is different from the closest method described in the VO.
    \item How did you choose hyperparameters (other design choices) and what are the values of the hyperparameters you chose for your final model? How did you make sure that the choice of hyperparameters works well?
  \end{itemize}
}

\subsection{Preprocessing}
\label{sec:preprocess}
Due to the imbalanced nature of the dataset, we employed an oversampling technique to balance the classes. We used a custom
oversampling function that generates synthetic samples for the minority class (fraudulent transactions) based on the k-nearest
neighbors algorithm. This approach involves generating new fraudulent samples by interpolating between existing fraudulent
samples and their nearest neighbors.
\subsubsection{Logistic Regression}

\subsubsection{Neural Network}
We used a neural network with 3 hidden layers and 1 output layer. The input layer has 30 neurons, and all hidden layers have 64 neurons. We used the ReLU activation function for all hidden layers and the sigmoid activation function for the output layer. We used the Adam optimizer and binary cross-entropy as the loss function. Furthermore, we trained the
model for 40 epochs with a batch size of 64. Because we implemented an early stopping mechanism, the model stopped at most after 24 epochs. We used a dropout layer with a dropout rate of 0.4 to prevent overfitting. For the learning rate, we started with 0.001 and decreased it by a factor of 0.1 if the validation loss did not decrease for 5 epochs.
For the early stopping mechanism, we used a patience value of 10 epochs. We used the roc-auc score as the evaluation metric for early stopping.

\subsubsection{Choice}
We chose the neural network over logistic regression because the neural network showed better results. The neural network was able to learn the underlying patterns in the data better than logistic regression.
See the \hyperref[sec:results]{Results'} section for more details.

\section{Results}
\label{sec:results}

{\color{blue}

  \begin{itemize}
    \item Describe the performance of your model (in terms of the metrics for your dataset) on the training and validation sets with the help of plots or/and tables.
    \item You must provide at least two separate visualizations
          (plot or tables) of different things, i.e. don’t use a table
          and a bar plot of the same metrics. At least three
          visualizations are required for the 3 person team.
  \end{itemize}
}

\section{Discussion}
\label{sec:discuss}

{\color{blue}
  \begin{itemize}
    \item Analyze the results presented in the report (comment on what contributed to the good or bad results). If your method does not work well, try to analyze why this is the case.
    \item Describe very briefly what you tried but did not keep for your final implementation (e.g. things you tried but that did not work, discarded ideas, etc.).
    \item How could you try to improve your results? What else would you want to try?

  \end{itemize}
}


\section{Conclusion}
\label{sec:con}

{\color{blue}

  \begin{itemize}
    \item Finally, describe the test-set performance you achieved. Do not
          optimize your method based on the test set performance!
    \item Write a 5-10 line paragraph describing the main takeaway of your project.
  \end{itemize}

}

%%%%%%%%%%%%%%%%%%%%%%%%%%%%%%%%%%%%%%%%%%%%%%%%%%%%%%%%%%%%%%%%%%%%%%%%%%%%%%%%



\end{document}
