%%%%%%%%%%%%%%%%%%%%%%%%%%%%%%%%%%%%%%%%%%%%%%%%%%%%%%%%%%%%%%%%%%%%%%%%%%%%%%%%
%2345678901234567890123456789012345678901234567890123456789012345678901234567890
%        1         2         3         4         5         6         7         8

\documentclass[a4, 10 pt, conference]{ieeeconf}  % Comment this line out if you need a4paper

%\documentclass[a4paper, 10pt, conference]{ieeeconf}      % Use this line for a4 paper

\IEEEoverridecommandlockouts                              % This command is only needed if 
                                                          % you want to use the \thanks command

\overrideIEEEmargins                                      % Needed to meet printer requirements.

% See the \addtolength command later in the file to balance the column lengths
% on the last page of the document

% The following packages can be found on http:\\www.ctan.org
%\usepackage{graphics} % for pdf, bitmapped graphics files
%\usepackage{epsfig} % for postscript graphics files
%\usepackage{mathptmx} % assumes new font selection scheme installed
%\usepackage{times} % assumes new font selection scheme installed
%\usepackage{amsmath} % assumes amsmath package installed
%\usepackage{amssymb}  % assumes amsmath package installed
\usepackage{multicol}
\usepackage{tcolorbox}
\usepackage{cuted,tcolorbox,lipsum}
\usepackage{xcolor}
\usepackage{hyperref}

\title{\LARGE \bf
Introduction to Machine Learning (SS 2024)\\ Programming Project
\vspace{-3em}
}


%\author{Fabio Plunser$^{1}$ and Dominik Barbist$^{2}$% <-this % stops a space
%}


\begin{document}


\maketitle
\vspace{-3em}
\thispagestyle{empty}
\pagestyle{empty}

\begin{strip}
  \begin{tcolorbox}[
      size=tight,
      colback=white,
      boxrule=0.2mm,
      left=3mm,right=3mm, top=3mm, bottom=1mm
    ]
    {\begin{multicols}{3}% replace 3 with 2 for 2 authors.

        \textbf{Author 1}       \\
        Last name: Plunser              \\  % Enter first name
        First name: Fabio             \\  % Enter first name
        Matrikel Nr.:               \\  % Enter Matrikel number

        \columnbreak

        \textbf{Author 2}       \\
        Last name: Barbist              \\  % Enter first name
        First name: Dominik             \\  % Enter first name
        Matrikel Nr.:               \\  % Enter Matrikel number

        \columnbreak

        % only four three person team
        % \textbf{Author 3}       \\
        % Last name:              \\  % Enter first name
        % First name:             \\  % Enter first name
        % Matrikel Nr.:               \\  % Enter Matrikel number

      \end{multicols}}
  \end{tcolorbox}
\end{strip}

%%%%%%%%%%%%%%%%%%%%%%%%%%%%%%%%%%%%%%%%%%%%%%%%%%%%%%%%%%%%%%%%%%%%%%%%%%%%%%%%

\section{Introduction}
\label{sec:intro}

We selected the Transaction dataset which includes ~22000 Data points each having 30 features(1 Time + 29 other features). The Data were labeled with a 1 if the transaction was fraudulent and 0 if it was not.
The dataset is highly imbalanced with only 0.xxx\% of the data points being labeled as fraudulent. The nature of the task is a classification problem either a transaction is fraudulent or not.
The 29 other features are anonymized, and we do not know what they represent. The dataset has no missing values and no data imbalances.
{\color{blue}
\begin{itemize}
  \item What is the nature of your task (regression/classification)? Is it about classifying types of birds, or deciding the number of cookies an employee receives?
  \item Describe the dataset (number of features, number of instances, types of features, missing data, data imbalances, or any other relevant information).
\end{itemize}
}

\section{Implementation / ML Process}
\label{sec:methods}

{\color{blue}

  \begin{itemize}
    \item Did you need to pre-process the dataset (e.g. augmenting data points, extracting features, reducing the dimensionality, etc.)? If so, describe how you did this.
    \item Specify the method (e.g. linear regression, or neural network, etc.). You do not have to describe the algorithm in detail, but rather the algorithm family and the properties of the algorithm within that family, e.g. which distance functions for a decision tree, what architecture (layers and activations) for a neural network, etc.
    \item State (in 2-5 lines) what makes the algorithm you chose suitable for this problem. What are the reasons for choosing your ML method over others?
    \item If you used a method that was not covered in the VO, describe how it is different from the closest method described in the VO.
    \item How did you choose hyperparameters (other design choices) and what are the values of the hyperparameters you chose for your final model? How did you make sure that the choice of hyperparameters works well?
  \end{itemize}
}

\subsection{Preprocessing}
\label{sec:preprocess}
Due to the imbalanced nature of the dataset, we used an oversampling technique to balance the dataset. Meaning we duplicated the fraudulent transactions to a certain percentage of the training data. Around 30\%-40\% showed the best results.
We tested two different methods, logistic regression, and a neural network because the data were labeled, and the task was a classification problem.
\subsubsection{Logistic Regression}

\subsubsection{Neural Network}
We used a neural network with 3 hidden layers and 1 output layer. The input layer has 30 neurons, and all hidden layers have 124 neurons. We used the ReLU activation function for all hidden layers and the sigmoid activation function for the output layer. We used the Adam optimizer and binary cross-entropy as the loss function. Furthermore, we trained the
model for 40 epochs with a batch size of 32. Because we implemented an early stopping mechanism, the model stopped at most after 18 epochs. We used a dropout layer with a dropout rate of 0.4 to prevent overfitting. For the learning rate, we started with 0.001 and decreased it by a factor of 0.1 if the validation loss did not decrease for 5 epochs.
For the early stopping mechanism, we used a patience value of 10 epochs. We used the F1 score as the evaluation metric because the dataset is imbalanced.

\subsubsection{Choice}
We chose the neural network over logistic regression because the neural network showed better results. The neural network was able to learn the underlying patterns in the data better than logistic regression.
See the \hyperref[sec:results]{Results'} section for more details.

\section{Results}
\label{sec:results}

{\color{blue}

  \begin{itemize}
    \item Describe the performance of your model (in terms of the metrics for your dataset) on the training and validation sets with the help of plots or/and tables.
    \item You must provide at least two separate visualizations
          (plot or tables) of different things, i.e. don’t use a table
          and a bar plot of the same metrics. At least three
          visualizations are required for the 3 person team.
  \end{itemize}
}

\section{Discussion}
\label{sec:discuss}

{\color{blue}
  \begin{itemize}
    \item Analyze the results presented in the report (comment on what contributed to the good or bad results). If your method does not work well, try to analyze why this is the case.
    \item Describe very briefly what you tried but did not keep for your final implementation (e.g. things you tried but that did not work, discarded ideas, etc.).
    \item How could you try to improve your results? What else would you want to try?

  \end{itemize}
}

\section{Conclusion}
\label{sec:con}

{\color{blue}

  \begin{itemize}
    \item Finally, describe the test-set performance you achieved. Do not
          optimize your method based on the test set performance!
    \item Write a 5-10 line paragraph describing the main takeaway of your project.
  \end{itemize}

}

%%%%%%%%%%%%%%%%%%%%%%%%%%%%%%%%%%%%%%%%%%%%%%%%%%%%%%%%%%%%%%%%%%%%%%%%%%%%%%%%



\end{document}
